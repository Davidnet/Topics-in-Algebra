\documentclass[11pt,a4paper]{article}
\usepackage[utf8]{inputenc}
\usepackage{amsmath}
\usepackage{amsfonts}
\usepackage{amssymb}
\usepackage{amsthm}
\author{David Cardozo}
\title{Topics in Algebra}
\newtheorem{define}{Definition}
\newtheorem{thm}{Theorem}
\newtheorem{claim}{Proposition}

\begin{document}
\maketitle

These are few notes and compilation of exercises of the book \textit{Topics in Algebra.} by I. N. Herstein.
\section{Preliminary notions}
\subsection{Set Theory}
Given a set $S$ we shall use the notation throughout $a \in S$ to read ``\emph{a is an element of S}''. The set $A$ will be said to be a \emph{subset} of $S$ if every element in $A$ is an element of $S$, We shall write $ A  \subset S$. Two sets $A$ and $B$ are equal, if both $ A  \subset B$ and $B \subset A$. A set $D$ will be called \emph{proper subset} of $S$ if $D \subset S$ but $ D \neq S$. The null set is the set having no elements; it is a subset of every set. Given a set $S$ we shall use the notation $A \left\lbrace a \in S \vert P(a)\right\rbrace $ to read `` $A$ is the set of all elements in $S$ for which the property $P$ holds.
\begin{define}
The \textbf{union} of the two sets $A$ and $B$, written as $A \cup B$, is the set $ \left\lbrace x \vert x \in A \text{ or } \in B \right\rbrace$
\end{define}
\textbf{Remark}  when we say that x is in A or x is in B, we mean x is in at least one of A or B, and may be in both.
\begin{define}
The \textbf{intersection} of the two sets $A$ and $B$, written as $ A \cap B$, is the set $ \left\lbrace x \vert x \in A \text{ and } x \in B \right\rbrace $

Two sets are said to be \emph{disjoint} if their intersection is empty.
\end{define}

\begin{claim}
For any three sets, $A,B,C$ we have
\begin{align*}
A \cap \left( B \cup C \right) = \left( A \cap B \right) \cup \left( A \cap C \right)
\end{align*}
\end{claim}
\begin{proof}
We will prove first $ \left( A \cap B \right) \cup \left( A \cap C \right) \subset A \cap \left( B \cup C \right) $. Observe $ B \subset B \cup C$, so that $ A \cap B \subset A \cap \left( B \cup C \right) $, in the same line of reasoning, $C \subset B \cap C$, so that $ A \cap C \subset A \cap \left( B \cap C \right) $, and we conclude $ \left( A \cap B \right) \cup \left( A \cap C \right) \subset \left( A \cap \left( B \cup C \right) \right) \cup \left( A \cap \left( B \cup C \right) \right) = A \cap \left( B \cup C \right) $.
Now for the other direction, let $x \in A \cap (B \cup C)$, so that $x \in A$, and $x \in B \cup C$; suppose the former, and we have that $ x \in ( A \cap B ) $. The second possibility, namely, $x \in C$, implies that $x \in A \cap C $. Thus in either case, $ x \in  (A \cap C) \cup (A \cap B)  $, whence $ A \cap \left( B \cup C \right) \subset \left( A \cap B \right) \left( A \cap C \right)$.
Combining the two concluding assertions, they give us the equality of both sets
\end{proof}

Given a set $T$, we say that $T$ serves as an \emph{index set} for the family $F = \left\langle A_{\alpha} \right\rbrace $ of sets if for every $ \alpha \in T$, there exist a set of $A_\alpha$ in the family $F$.
By the iunion of the set $A_\alpha$, where $\alpha$ is in $T$, we mean the set $ \left\lbrace x \vert x \in A_\alpha \text{ for  at least one } \alpha \in T \right\rbrace$. We denote it by $ \cup_{\alpha \in T} A_{\alpha} $. Similarly, we denote the intersection of the sets $A_\alpha$ by $ \cap_{\alpha \in T} A_{\alpha} $. The sets $A_\alpha$ are mutually disjoint if for $ \alpha \neq \beta$, $ A_\alpha \cap A_\beta $ is the null set.
\begin{define}
Given the two sets A, B then the 
\textbf{difference set}, $A - B$, is the set $ \left\lbrace x \in A \vert x \not\in N \right\rbrace$
\end{define}
\begin{claim}
For any set $B$, the set $A$ satisfies  
\begin{align*}
A = ( A \cap B) \cup (A - B). 
\end{align*}
\end{claim}
\begin{proof}
Again, using the same strategy used before, we will show first $ (A \cap B) \cup (A - B) \subset A $, Observe that $A \cap B \subset A$, and $ A - B \subset A$ so that $ ( A \cap B ) \cup ( A - B ) \subset A $
Now, for the converse, we want to see $ A  \subset (A \cap B)$, first observe that if we suppose that $x \in A$, then either $ x \in (A-B)$ or $x \in A \cap B$, so that in either case, eventually $x \in ((A \cap B) \cup (A - B) )$, so we conclude $ A  \subset (A \cap B) $. Finally, combining the concluding assertions we have $ A = (A \cap B) $ which proves our proposition.
\end{proof}

Observe $B \cap (A - B) $ is the null set. Observe than when $ B \subset A $, we call $A-B$ the complement of $B$ in $A$.

\begin{define}
	The binary relation $ \sim $ on $ A $ is said to be an \emph{equivalence relation} on $A$ if for all $a,b,c$ in $A$
	\begin{itemize}
		\item $a \sim a$ 
		\item $a \sim b $ implies $ b \sim a $ 
		\item $ a \sim b $ and $ b \sim c$ imply $a \sim c$
	\end{itemize}
	The first of these properties is called reflexivity, the second, symmetry, and the third, transitivity
\end{define}

\begin{define}
	If $A$ is a set and if $\sim$ is an equivalence relation on $A$, then the equivalence class of $a \in A$ is the set $ \left\lbrace x \in A \vert a \sim x \right\rbrace.$ We write it as cl(a).
\end{define}

Now it comes the first big result.
\begin{thm}
	The distinct equivalence classes of an equivalence relation on $A$ provide us with a decomposition of $A$ as a union of mutually disjoint subsets. Conversely, given a decomposition of $A$ as a union of mutually disjoint, nonempty subsets, we can define an equivalence relation on $A$ for which these subsets are the distinct equivalence classes.
\end{thm}

\begin{proof}
	Let the equivalence relation on $A$ denoted by $ \sim $. Observe first that $ a \sim s$, so that $ a \in $ cl($a$), whence the union of all cl($a$)'s is all of $A$. We will now prove that two equivalence classes are either equal or disjoint, so suppose for the contrary that two distinct classes cl(a) and cl(b) their intersection is nonempty; then there exist an element $ x \in $ cl(a) and $ x \in $ cl(b), that is, $ x \sim a$ and $ x \sim b$, and by transitivity property of the equivalence relationship, we have $ a \sim b$, now let $ y \in $ cl(b); thus we have $ b \sim y$. But, from $ a \sim b$, and $ b \sim y$, we have then $ a \sim y$, so that, $ y \in $ cl(a), and we conclude cl(b) $ \subset$ cl(a), we observe also that the argument for $ y \in $ cl(a) is symmetric, so that cl(a) $ \subset$ cl(b), and we have the contradiction that we took two distinct classes, but cl(a) $=$ cl(b). We conclude then that the distinct cl(a)'s are mutually disjoint and their union is $A$. Now for the other part of the theorem. \par
	Suppose that $ A = \cup A_{\alpha} $, where the $a_{\alpha} $ are mutually disjoint, nonempty sets. We define an equivalence relation $ \sim $ as: given $a \in A$ (since $a$ is in exactly one of the $A_{\alpha}$ ), we define $a \sim b$ if and only if $a$ and $b$ are in the same $A_{\alpha}$, we need to check if $ \sim $ is an equivalence relation. First, observe $ a \sim a$ since, again $a$ is in exactly one of the $A_{\alpha}$. Second, suppose $ a \sim b$, that is $a,b \in A_{\alpha}$ for some unique $ \alpha $, which is the same as $ b \sim a$. Finally, suppose $a \sim b$, and $ b \sim c$ so that $a,b,c \in A{\alpha}$ for some unique $\alpha$, and we can see that $a \sim c$. \par 
	Finally, let us observe that for any $a \in A$, cl(a) is a subset of $A$, so that cl(a) $\subset$ A, and $\bigcup_{\alpha \in A} cl(a) \subset A$; and let for $b \in A$, there exist a unique set cl(b) up to equivalence classes, so that $ A \subset \bigcup_{\alpha \in A} cl(\alpha) $ 
\end{proof}
\subsection{Problems of Set Theory}
\textbf{5.} For a finite set $C$ let $ \vert C \vert$ indicate the number of elements in $C$. If $A$ and $B$ are finite sets prove $ \vert A \cup B \vert = \vert A \vert + \vert B \vert - \vert A \cap B \vert $ \par
\textbf{Solution}
Suppose $A$ and $B$ are finite sets, so that there exist $n,m \in \mathbb{N}$ for which $ \vert A \vert = n$, and $ \vert B \vert = m$. Let us remark that if $D, C$ are finite set which are disjoint, $ \vert D \cup C \vert $ is $ \vert D \vert + \vert C \vert $. Given this two facts, observe $ A \cup B = (A - (A \cap B)) \cup B$, and $ A - (A \cap B) \cap B = \emptyset $, so that $ \vert A \cup B \vert = \vert A \vert - \vert A \cap B \vert + \vert B \vert $. \par
\textbf{6.} If $A$ is a finite set having $n$ elements, prove that $A$ has exactly $ 2^{n} $ distinct subsets. \par 
\textbf{Solution}
The proof is by induction. First, observe that if a set $A$ has one element, the subset of $A$ are $ \left\lbrace \emptyset, A \right\rbrace $ which has $ 2^{1} $ elements, so that the assertion is true for $ n = 1 $, now suppose that if a set $A$ has $n$ elements, then there are exactly $2^{n}$ distinct subsets. Now consider the set with $n+1$ elements given by $ \left\lbrace a \right\rbrace \cup A $, with $ a \not\in A $. So that the subsets of $ \left\lbrace a \right\rbrace \cup A $ are given by first taking the subsets of $A$, which we know have $2^{n}$ elements, and then taking a copy of these subset and adding the element $ a $, so that there exist:
\begin{align*}
2^n + 2^n =2^n (1 + 1) =2^n2 = 2^{n+1}
\end{align*} 
so that for a set with $n+1$ element, there are exactly $ 2^{n+1} $ subsets. Then, by the principle of mathematical induction, we have shown that If $A$ is a finite set having $n$ elements, prove that $A$ has exactly $ 2^{n} $ distinct subsets. \par
\textbf{10} Let $S$ be a set and let $S^*$ be the set whose elements are the various subsets of $S$. In $S^*$ we define an addition and multiplication as follows: If $A,B \in S^*$:
\begin{itemize}
	\item $ A + B = (A-B) \cup (B-A)$
	\item $ A \cdot B = A \cap B $
\end{itemize}
Prove the following laws: \par 
\begin{align*}
(A + B) + C = A + (B + C)
\end{align*}
\textbf{Proof} Observe that  $ x \in A + B$ if and only if, $ x \not\in A \cap B $, so that $ x \in ((A+B) + C) $ if and only if $ x \not\in (A+B) \cap C$, or in other words, $ x \not\in (A \cap B \cap C)$, which by the property that $\cap$ is associative, we have then $ x \in (A + (B+C)) $. So we conclude then $ (A + B) + C = A + (B + C) $.
\begin{align*}
A \cdot (B + C) = A \cdot B + A \cdot C
\end{align*}
\textbf{Proof} Observe:
\begin{align*}
	 A \cdot B + A \cdot C = A \cap B + A \cap C \\
	 A \cap B + A \cap C = (A \cap B - A \cap C) \cup (A \cap C - A \cap B) \\
	 (A \cap B - A \cap C) \cup (A \cap C - A \cap B)  = A \cap (B-C) \cup A \cap (C-B) \\
	 A \cap (B-C) \cup A \cap (C-B) = A \cap ((B-C) \cup (C-B)) \\
	 A \cap (B-C) \cup A \cap (C-B) = A \cdot (B+C)
\end{align*} 
\par
\begin{align*}
A \cdot A = A
\end{align*}
\textbf{Proof}
By definition. $A \cdot A = A \cap A = A $
\par
\begin{align*}
A + A = \emptyset
\end{align*}
\textbf{Proof}
By definition. $A+A = (A - A) \cup (A - A) = \emptyset \cup \emptyset = \emptyset$
\par 
If $ A + B = A + C$, then $ B = C$
\textbf{Proof}
Suppose is false that $B = C $, without loss of generalization, say that  $x \in B$, but $ x \not\in C$. So two cases can happen:
\begin{itemize}
	\item[i)] if $x \in A + B$, observe that $ x \not\in A$. Note that $x \in A + B $ is equivalent to $ x \in A + C$, so that $ x \in C$. A contradiction.
	\item[ii)] if $x \not\in A + B$, we have that $x \not\in A + C$, so that $ x \in A \cap C$, which implies again $ x \in C$. A contradiction.
\end{itemize}
We conclude then, if $ A + B = A + C$, then $ B = C$.
(The system just described is an example of a\textit{ Boolean Algebra}.)
\textbf{12.} Let $S$ be the set of all integers an let $ n >1$ be a fixed integer. Define for $a,b \in S$, $a \sim b$ if $ a - b$ is a multiple of $n$. 
\begin{claim}
	$\sim$ is an equivalence relation
\end{claim}
\textbf{Proof}
First, observe that $ a \sim a$ since $ a - a = 0$ ad $0 \cdot n = 0$, second, suppose $ a \sim b$, or in other words, $ kn = a -b$ for some integer $k$, observe $ -kn = b - a$, and we have then $ b \sim a$. Finally, suppose $ a \sim b$ and $b \sim c$, more explicitly, $kn = a -b$ and $gn = b -c$ for some integers $k$ and $g$, take note that $kn + gn = (k +g)n = a -c$, so that $ a \sim c$. \par 
\begin{claim}
	There are exactly $n$ distinct classes
\end{claim}
\textbf{Proof}
Let cl(0), ...,cl(n-1), be the $n$ different equivalence classes, defined by $ \sim $ as above, observe that for an integer $ x \geq n $, cl(x) is: $ \left\lbrace m \in \mathbb{Z} \vert x \sim m \right\rbrace $, note that  if $ x \geq n $ is equivalent to say that, there exist integers  $ E = {0,1,...} $ and $K = {0,...,n-1} $ such that $ x = En + K $. Since $ x \sim m $, $ En + K \sim m$, which by definition is: there exist an integer $R$ for which $ Rn= (En + K) - m$, or $ (R - E)n = K - m $, so that $ K \sim m$, and since $K = {0,...,n-1}$, w have shown for $ x \geq n $, $x$ is in the any of the equivalence classes of cl(0), cl(1), \ldots cl($n-1$).
\subsection{Mappings}
We introduce the concept of a mapping of one set into another. Informally, a mapping from one set, $S$, into another, $T$, is a ruler that associates with each element in $S$ a \emph{unique} element $t$ in $T$.
\begin{define}
	If $S$ and $T$ are nonempty sets, then a \textbf{mapping} from S to T is a subset, M, of $ S \times T$ such that for every $ s \in S$ there is a unique $t \in T$ such that the ordered pair $ (s,t)$ is in M.	
\end{define}
Alternatively and for pedagogical reasons, we think of a mapping as a rule that associates any element $s \in S$ some element $ t \in T$. We shall say that $t$ is the image of $s$ under the mapping.
\textbf{Notation Remarks} Let $ \sigma $ be a mapping from $S$ to $T$; we denote this by writing $\sigma : S \rightarrow T$ or $ S \xrightarrow{\sigma} T$, strangely enough, if $t$ is the image of $s$ under $\sigma$ we shall represent this fact by $ t = s\sigma $. Algebraists often write mappings on the right. 

Given a mapping $ \tau: S \rightarrow T$ we define for $t \in T$, the inverse image of $t$ with respect to $ \tau$ to be the set $ \left\lbrace s \in S \vert t = s\tau \right\rbrace $

\begin{define}
	The mapping $ \tau $ of $S$ into $T$ is said to be \textbf{onto} $T$ if given $ t\in T$ there exits an element $ s \in S $ such that $ t = s\tau $
\end{define}
Observe that we call the subset $S\tau = \left\lbrace x \in T \vert x = s\tau \text{ for some  } s \in S \right\rbrace$ the \textbf{image} of $S$ under $ \tau $, then $ \tau $ is onto if the image of $S$ under $ \tau $ is all of $T$.

\begin{define}
	The mapping $ \tau $ of $S$ into $T$ is said to be a \textbf{one-to-one mapping} if whenever $ s_1 \neq s_2 $, then $ s_1\tau = s_2\tau$.
\end{define}
\textbf{Remark} Observe that the mapping $ \tau$ is one-to-one if for any $ t \in T$ the inverse image of $t$ is either empty or is a set consisting of one element.

\begin{define}
	The two mappings $ \sigma $ and $\tau$ of $S$ into $T$ are said to be \textbf{equal} if $s\sigma= s\tau $ for every $ s \in S$.
\end{define}

\begin{define}
	If $\sigma: S \rightarrow T$ and $\tau: T \rightarrow U$ then the \textbf{composition} of $ \sigma $ and $\tau$ (also called their product) is the mapping $ \sigma \circ \tau: S \rightarrow U$ defined by means of $s( \sigma \circ \tau) = (s\sigma)\tau $ for every $ s \in S$
\end{define}
Remark that it is read from left to right; that is, $ \sigma \circ \tau$: first perform $ \sigma $ and then do $ \tau$. \par 

For mappings of sets, provided the products make sense, the following holds:

\begin{claim}[Associative Law] \label{pro:Associativelawmaps}
	If $ \sigma: S \rightarrow T$, $\tau: T \rightarrow U$, and $ \mu: U \rightarrow V$, then $ (\sigma \circ \tau) \circ \mu = \sigma \circ ( \tau \circ \mu).$
\end{claim}

\begin{proof}
	So, lt us see that the composition makes sense to be equal, observe that $ \sigma \circ \tau$ takes an element of $ s \in S $ to $ t \in T$ so that $ (\sigma \circ \tau) \circ \mu $ make sense and takes $ S $ into $V$. Similarly $ \tau \circ \mu$ takes $T$ into $V$ and $ \sigma \circ (\tau \circ \mu) $ takes also $S$ into $V$, so we are rest to check if both functions are equal.
	So, we start with an element $s$ in $S$ and we want to see that $ s((\sigma \circ \tau) \circ \mu) = s(\sigma \circ (\tau \circ \mu))$. \par 
	By definition of the composition of maps $s((\sigma \circ \tau) = (s(\sigma \circ \tau))\mu = ((s\sigma)\tau)\mu$, whereas $ s(\sigma \circ (\tau \circ \mu)) = (s \sigma)(\tau \circ \mu) = (((s \sigma)\tau)\mu $. Thus, we conclude $ (\sigma \circ \tau) \circ \mu = \sigma \circ ( \tau \circ \mu). $ 
\end{proof}

\begin{claim} \label{pro:propertystillcomposition}
	Let $\sigma: S \rightarrow T$ and $ \tau: T \rightarrow U$; then
	\begin{itemize}
		\item $\sigma \circ \tau $ is \textbf{onto} if each of $\sigma$ and $\tau$ is onto.
		\item $\sigma \circ \tau $ is \textbf{one-to-one} if each of $\sigma$ and $\tau$ is one-to-one. 
	\end{itemize}
\end{claim}
\begin{proof} \par 
	\begin{itemize}
		\item Suppose $ \sigma $, and $ \tau$ both are onto functions, that is, $ S\sigma = T $, and $T\tau = U$, so we establish that $(S\sigma)\tau = U$, or equivalently, $ S (\sigma \circ \tau) = U$, and we have then $\sigma \circ \tau $  is onto.
		\item Suppose that $s_1,s_2$ ar elements of $S$, and that $ s_1 \neq s_2$. By the one-to-one property of $ \sigma$, $s_1 \sigma \neq s_2 \sigma$. Since $ \tau $ is also one-to-one, and $ s_1 \sigma , s_2 \sigma $ are distinct elements of $T$, $ (s_1 \sigma)\tau \neq (s_2 \sigma)\tau $, therefore $ s_1(\sigma \circ \tau) = (s_1 \sigma) \tau \neq (s_2 \sigma)\tau = s_2(\sigma \circ \tau)$, and we establish the lemma.
	\end{itemize}
\end{proof}

Now, let us suppose that $ \sigma $ is a one-to-one mapping of $S$ \emph{onto} $T$; we call $\sigma$ a \textbf{one-to-one correspondence}
between $S$ and $T$. so observe that for any $t \in T$, there exist $ s \in S$ such that $t = s\sigma $; and by the property of one-to-one, this $s$ is unique. We define the mapping $ \sigma^{-1}: T \rightarrow S $ by $ s = t \sigma^{-1} $ if and only if $ t = s\sigma $. This map, is called the \textbf{inverse} of $ \sigma$. Observe $ \sigma \circ \sigma^{-1} $ maps $S$ to $S$, note that $ s(\sigma \circ \sigma^{-1}) = (s \sigma)\sigma^{-1} = t \sigma^{-1} = s $. so that $ \sigma \circ \sigma^{-1} $ is the identity operator on $S$, similarly $ \sigma^{-1} \circ \sigma $ is again the identity mapping of $T$.
Conversely, if $\sigma: S \rightarrow T$  is such that there exist $ \mu: T \rightarrow S$ with the property that $ \sigma \circ \mu$ and $ \mu \circ \sigma$ are the identity mappings on $S$ and $T$ respectively, we claim that $\sigma$ is a one-to-one correspondence between $S$ and $T$. First, observe that $ \sigma$ is onto, this given by the fact that, let $t \in T$, so that = $t = t I_t = t(\mu \circ \sigma) = (t\mu)\sigma$, and we take note that $ t \mu \in S$ so that we shown that there exist $ s \in S$ such that $ t = s \sigma $ for any $t$ and we conclude, $ \sigma $ is onto. Second, suppose $ s_1 \sigma = s_2 \sigma $; take note that $ s_1 = s_1 I_s = s_1 (\sigma \circ \mu) = ( s_1 \sigma)\mu = (s_2 \sigma)\mu = s_2(\sigma \circ \mu) = s_2 I_s = s_2 $
The preceding discussion proves then:
\begin{claim} \label{pro:existenceofonetoonecorrespondence}
	The mapping $ \sigma: S \rightarrow T $ is a \textbf{one-to-one correspondence} between $S$ and $T$ if and only if there exist a mapping $\mu: T \rightarrow S$ such that $ \sigma \circ \mu $ and $ \mu \circ \sigma $ are the identity mappings on $S$ and $T$ respectively.
\end{claim}

\begin{define}
	If $S$ is a nonempty set then $A(S)$ is the \textbf{set of all one-to-one mappings} of $S$ onto itself.
\end{define}
$A(S)$ plays a universal type of role in groups.
We state the following theorem as the recollection of th previous lemma and results:
\begin{thm}
	If $ \sigma, \tau, \mu $ are elements of $ A(S) $, then:
	\begin{itemize}
		\item $ \sigma \circ \tau $ is in $ A(S) $.
		\item ($\sigma \circ \tau) \circ \mu = \sigma \circ (\tau \circ \mu)$.
		\item There exist an element $i$ (the identity map) in $A(S)$ such that $ \sigma \circ i = i \circ \sigma = \sigma$.
		\item There exist an element $ \sigma^{-1} \in A(S)$ such that $ \sigma \circ \sigma^{-1} = \sigma^{-1} \circ \sigma = i$.
	\end{itemize}
\end{thm}
\begin{proof}
	\begin{itemize}
		\item Given by Proposition \ref{pro:propertystillcomposition}
		\item Given by Proposition \ref{pro:Associativelawmaps}
		\it	em Given by Proposition \ref{pro:existenceofonetoonecorrespondence}
		\item Given by Proposition \ref{pro:existenceofonetoonecorrespondence}
	\end{itemize}
\end{proof}

\begin{claim}
	If $S$ has more than two elements we can find two elements $\sigma, \tau$ in $A(S)$ such that $ \sigma \circ \tau \neq \tau \circ \sigma$.
\end{claim}

\begin{proof}
	Consider $U$ subset of $S$, with $S$ having more than three elements, let us also suppose that the cardinality of $U$ is $3$. Define $T$ as the complement of U with respect to S, that is, $T = S - U$. Now consider $F$ the set of all one-to-one mappings and onto functions on $S$ that fix $T$, i.e if $f \in F$ then $f(t) = t $ for all $t \in T$. Observe then that all functions on $F$ are uniquely determined by the values of $f \vert_{U} : S \rightarrow S$. Let us rename the elements of $U$ with $x_1,x_2,x_3$, define the mapping $\sigma \in F$ by $\sigma: S \rightarrow S$ by $x_1 \sigma = x_2$, $x_2 \sigma = x_3$, $x_3 \sigma = x_1$, and the rest is determined already since $ \sigma \in F$. Define the mapping $ \tau: S \rightarrow S$, and $ \tau \in F$ by $x_2 \tau = x_3$, $x_3\tau = x_2$ and $ x_1 \tau = x_1$. Take note, $x_1(\sigma \circ \tau) = x_3$ but that  $ x_1(\tau \circ \sigma) = x_2 \neq x_3$. Thus $ \sigma \circ \tau \neq \tau \circ \sigma$ 
\end{proof}

\subsection{Problems of Mappings}
\textbf{2} If $S$ and $T$ are nonempty sets, prove that there exists a correspondence between $ S \times T$ and $ T \times S$


\textbf{Proof}
Let us define the following mappings:
\begin{align*}
\sigma: (S \times T) \rightarrow (T \times S) \\
(s,t) \longmapsto (t,s)
\end{align*}
and 
\begin{align*}
\mu: (T \times S) \rightarrow (S \times T) \\
(t,s) \longmapsto (s,t)
\end{align*}
Observe first that, $ \sigma \circ \mu  = i_{(S \times T)} $, since $ (s,t)(\sigma \circ \mu) = ((s,t)\sigma)\mu = (t,s)\mu = (s,t) = (s,t)i_{(S \times T)} $, similarly for $ \mu \circ \sigma = i_{(T \times S)} $





\end{document}
